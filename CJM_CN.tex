%!TEX program = xelatex
% XeLaTex
% document type
\documentclass{ctexart}

% preamble
% - packages
\usepackage{array, titlesec, calc, fancyhdr}
\usepackage[margin=3cm]{geometry}
\usepackage[colorlinks=true, urlcolor=blue]{hyperref}

% - commands

\newcolumntype{L}{>{\raggedleft\arraybackslash}p{2cm}}
\newcolumntype{R}{>{\raggedleft\arraybackslash}p{3.5cm}}
\newcommand{\parlength}{\textwidth-2.5cm}

\fancypagestyle{onestyle}
{
    \fancyhead{}
    \fancyfoot[CO]{\thepage}
    \renewcommand{\headrulewidth}{0pt}
    \renewcommand{\footrulewidth}{0pt}
}

\titleformat
    {\section}
    {\normalfont\LARGE\bfseries}
    {\thesection}
    {1em}
    {}
    [{\titlerule[0.7pt]}]

\pagestyle{onestyle}

% - title
\author{}
\title{\vspace{-1cm}曹建明}
\date{}

% document
\begin{document}
% \maketitle


% - personal info
\section*{个人信息}

\noindent\begin{tabular}{rl}
姓名: & 曹建明\\
手机: & +8615505139159\\
出生年月:& 1987/02\\
教育背景:& 统招本科,南京工业大学(2005-2009),建筑环境与设备工程专业\\
邮箱: & \href{mailto:glide.ming@gmail.com}{glide.ming@gmail.com}\\
获取最新简历: & \href{https://github.com/SimpleExpress/resume/blob/master/CJM_CN.tex}{https://github.com/SimpleExpress/resume}
\end{tabular}

% - Skill
\section*{职业能力}

\begin{itemize}
\item 6年大数据开发经验,3年小团队管理经验,拥有丰富的数据开发及治理经验;4年软件测试经验,善于工程质量保障。
\item 熟悉主流大数据生态组件及原理,如Hadoop、Hive、Spark、Flink、ClickHouse、Druid、ZooKeeper、Kafka、Redis/Codis、MySQL、数据湖	等。
\item 拥有较强的编码能力及自驱学习能力,熟练使用Java/Scala/Python/Go,二次开发过Superset、Druid、chpxory,阅读过Spark、Hadoop、Kafka部分源码,了解常见的分布式协议如Raft算法、Quorum NWR算法、ZAB协议等,熟悉hashicorp/raft源码及其基本使用。
\item 优秀的理解沟通能力,能快速理解业务背景,对数据敏感。
\end{itemize}

% - work experience
\section*{工作经验}

\noindent\begin{tabular}{L|l}
2014.10. -- 今 & 
\parbox
    {\parlength}
    {
        {\bf 上海触宝科技,增长技术专家} \\
        曾负责国内业务数据治理工作,从无到有搭建了国内业务数据团队,助力数字化转型;后负责增长核心服务及数据建设,成功助力疯读小说一年步入免费小说前三,国内全产品DAU增长超10倍,在增长领域进行了深度实践。
        \begin{itemize}
            \item Android架构(半年):1)负责Android客户端后台任务优化,开发实现了任务调度框架,支持灵活的调度方式及执行策略。2)负责重构Android客户端的双卡自动识别引擎,调研了超过20个不同品牌上百种机型,实现了基于自适应探查加云数据修正的自动识别模型,能够自动识别市面上超过10000种不同厂商与型号的机型,在当时做到了业内领先。
            \item 国内数据团队组建及数据治理:1)从零搭建数据处理管线,构建数据处理框架,联动MR、Pig、Spark、Sqoop、Azkaban、Druid、Drill等,大大提升团队的开发效率。2)优化移动客户端数据埋点sdk并加以推广,基于Kimball维度建模思想对业务数据进行全面梳理,使用Apache Drill+Parquet构建业务数仓,围绕业务分析多次组织数据分享,提升业务同学的数据使用效能,对于业务的决策起到重要的支撑作用。3)招聘并组建7人的数据团队。
            \item 增长架构:1)基于Go+Codis+MySQL+Tablestore+Kafka构建高可用归因服务,主导归因服务产品及技术设计,并深入业务探索归因策略,成功挖掘多种有效策略提升投放效果。2)基于Spark+Hive+Kafka+Flink构建用户生命周期(LTV/ROI)数据及实时策略数据链路,极大地提升了业务决策能效,是国内增长的核心数据能力。3)基于Go+Codis+Flink+Kafka搭建RTA服务,设计并实现RTA并行实验能力、使用Python+Flask+MySQL搭建RTA策略平台、完成了RTA策略的制定-测试-分析-监控的闭环;主导RTA策略的挖掘工作,成功挖掘出多个策略并运用在多个产品的投放上,整体投放效果提升明显。4)参与数据平台的建设以提升增长数据能力,期间主要负责的工作有:Druid的部署运维及定制开发工作;Clickhouse的部署运维及代理chproxy的定制开发工作;基于Flink/Spark扩展更多数据源的支持,并基于Java/Scala+Spring搭建流平台,支持SQL化的批流一体处理平台,构建流平台+Clickhouse的实时数仓体系。
        \end{itemize}

    }
\end{tabular}\\\\

\noindent\begin{tabular}{L|l}
2010.7. -- 2014.9. & 
\parbox
    {\parlength}
    {
        {\bf 资深软件测试工程师} \\
        在NCS, IBM从事软件测试工作,拥有丰富的软件测试经验。在NCS期间曾两次获得客户颁发优质贡献证书。
        \begin{itemize}
        \item 参与或主导测试项目,涉及测试规划,测试设计,测试执行,确保测试项目的质量周期完善可控。
        \item 拥有丰富的自动化与性能测试经验,参与过自动化测试框架的设计与编写,拥有良好的编码能力和技术调研能力,通过为客户提供高度针对性的方案与演示,多次帮助公司与客户达成深度合作。
        \end{itemize}
    }
\end{tabular}\\\\

\end{document}