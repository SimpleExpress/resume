%!TEX program = xelatex
% XeLaTex
% document type
\documentclass{ctexart}

% preamble
% - packages
\usepackage{array, titlesec, calc, fancyhdr}
\usepackage[margin=3cm]{geometry}
\usepackage[colorlinks=true, urlcolor=blue]{hyperref}

% - commands

\newcolumntype{L}{>{\raggedleft\arraybackslash}p{2cm}}
\newcolumntype{R}{>{\raggedleft\arraybackslash}p{3.5cm}}
\newcommand{\parlength}{\textwidth-2.5cm}

\fancypagestyle{onestyle}
{
    \fancyhead{}
    \fancyfoot[CO]{\thepage}
    \renewcommand{\headrulewidth}{0pt}
    \renewcommand{\footrulewidth}{0pt}
}

\titleformat
    {\section}
    {\normalfont\LARGE\bfseries}
    {\thesection}
    {1em}
    {}
    [{\titlerule[0.7pt]}]

\pagestyle{onestyle}

% - title
\author{}
\title{\vspace{-1cm}曹建明}
\date{}

% document
\begin{document}
% \maketitle


% - personal info
\section*{个人信息}

\noindent\begin{tabular}{rl}
姓名: & 曹建明\\
手机: & +8615505139159\\
出生年月:& 1987/02\\
教育背景:& 统招本科,南京工业大学\\
邮箱: & \href{mailto:glide.ming@gmail.com}{glide.ming@gmail.com}
\end{tabular}

% - Skill
\section*{职业能力}

\begin{itemize}
\item 6年大数据开发经验,3年数据小团队管理经验,拥有丰富的数据开发及治理经验;4年软件测试经验,善于工程质量保障。
\item 熟悉主流大数据生态组件及原理,如Hadoop、Hive、Spark、Flink、ClickHouse、Druid、ZooKeeper、Kafka、Redis/Codis、MySQL、数据湖	等。
\item 拥有较强的编码能力及自驱学习能力,熟练使用Java/Scala/Python/Go,二次开发过Superset、Druid、chpxory,阅读过Spark、Hadoop、Kafka部分源码,了解常见的分布式协议如Raft算法、Quorum NWR算法、ZAB协议等,熟悉hashicorp/raft源码及其基本使用。
\item 优秀的理解沟通能力,能快速理解业务背景,对数据敏感。
\end{itemize}

% - work experience
\section*{工作经验}

\noindent\begin{tabular}{L|l}
2014.10. -- 今 & 
\parbox
    {\parlength}
    {
        {\bf 上海触宝科技,增长技术专家} \\
        曾负责国内业务数据的治理工作,从无到有搭建了国内业务数据团队;后负责增长核心服务及数据建设,在增长领域进行了深度实践。
        \begin{itemize}
            \item 在触宝电话Android架构组工作了近半年时间,主要产出为:1)主导了Android客户端的后台耗电优化工作。在此过程中基于AlarmManager实现了后台任务调度框架建设,形成一个统一管理的,支持灵活策略(如充电检测和WiFi检测运行)的任务调度框架;2)独立重构了Android客户端的双卡识别引擎,期间调研了超过20个不同品牌上百种机型,建立并开发了数据驱动的自动识别模型及云数据同步服务,该模型能够自动识别市面上超过10000种不同厂商与型号的机型,在当时做到了业内领先。
            \item 负责国内业务数据治理工作,主要产出为:1)从零开始搭建了数据处理技术链路,包括MR、Pig、Spark、Sqoop、Azkaban等技术组件调研与推广,编写了大量的数据处理框架性质的代码,对Azkaban进行扩展以支持微信、电话等报警能力,大大提升团队的开发效率。2)和业务部分紧密沟通,支持了大量业务数据报表的开发需求,探索基于Caravel(现为Apache Superset)+Druid的可视化方案,参与了部分Caravel的二次开发工作。3)数据治理与数仓建设,对数据打点能力做了进一步地封装,同时组织了多次数据打点规范的分享,优化了客户端的数据打点,有效提升了数据处理效率;使用Kimball维度建模思想对业务数据进行全面梳理,基于Apache Drill构建业务数仓,进一步规范了数据处理标准,同时也大大提升了数据分析同学的效率,对于业务的决策起到重要的辅助作用。
            \item 18年底,在国内业务疲软之际,开始接触增长相关的工作,负责国内增长的核心技术体系搭建,主要产出为:1)作为归因服务负责人,和投放同学及各媒体紧密沟通,持续产出优化方案对归因服务进行迭代演进;目前已经接入了超过10家媒体,同时基于对投放的认知,成功挖掘出了多个实时归因策略,有效优化了投放效果。2)围绕投放增长的模式,构建以刻画用户粒度的来源、获客成本、留存、运营成本、收入贡献等生命周期(LTV/ROI)核心数据,极大地提升了业务决策能效。3)接手RTA服务,对服务进行了技术优化与功能升级,实现了并行实验能力、数据标准化、实验配置web化等重大改进,形成了RTA策略的制定-测试-分析-监控的闭环,大大提升了RTA服务的业务贡献;主导RTA策略的挖掘工作,成功挖掘出多个策略并运用在多个产品的投放上,整体投放效果提升明显。
        \end{itemize}

    }
\end{tabular}\\\\

\noindent\begin{tabular}{L|l}
2010.7. -- 2014.9. & 
\parbox
    {\parlength}
    {
        {\bf 资深软件测试工程师} \\
        在NCS, IBM从事软件测试工作,拥有丰富的软件测试经验。在NCS期间曾两次获得客户颁发优质贡献证书。
        \begin{itemize}
        \item 参与或主导测试项目,涉及测试规划,测试设计,测试执行,确保测试项目的质量周期完善可控。
        \item 拥有丰富的自动化与性能测试经验,参与过自动化测试框架的设计与编写,拥有良好的编码能力和技术调研能力,通过为客户提供高度针对性的方案与演示,多次帮助公司与客户达成深度合作。
        \end{itemize}
    }
\end{tabular}\\\\

\end{document}