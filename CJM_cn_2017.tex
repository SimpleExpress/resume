% XeLaTex
% document type
\documentclass{ctexart}

% preamble
% - packages
\usepackage{array, titlesec, calc, fancyhdr}
\usepackage[margin=3cm]{geometry}
\usepackage[colorlinks=true, urlcolor=blue]{hyperref}

% - commands

\newcolumntype{L}{>{\raggedleft\arraybackslash}p{2cm}}
\newcolumntype{R}{>{\raggedleft\arraybackslash}p{3.5cm}}
\newcommand{\parlength}{\textwidth-2.5cm}

\fancypagestyle{onestyle}
{
    \fancyhead{}
    \fancyfoot[CO]{\thepage}
    \renewcommand{\headrulewidth}{0pt}
    \renewcommand{\footrulewidth}{0pt}
}

\titleformat
    {\section}
    {\normalfont\LARGE\bfseries}
    {\thesection}
    {1em}
    {}
    [{\titlerule[0.7pt]}]

\pagestyle{onestyle}

% - title
\author{}
\title{\vspace{-1cm}曹建明}
\date{}

% document
\begin{document}
% \maketitle


% - personal info
\section*{个人信息}

\noindent\begin{tabular}{rl}
姓名: & 曹建明\\
手机: & +86 155 0513 9159\\
邮箱: & \href{mailto:glide.ming@gmail.com}{glide.ming@gmail.com}\\
{\sc GitHub:} & \href{https://github.com/SimpleExpress}{https://github.com/SimpleExpress}
\end{tabular}

% - Skill
\section*{职业技能}

\begin{itemize}
\item 有较强的沟通协调能力,学习调研与独立开发能力,能够独立推动项目顺利进行。
\item 有良好的算法基础与代码风格,超过5年Python和Java使用经验, 熟悉Scala编程语言。
\item 熟练编写MapReduce, Pig, Spark任务代码及其性能调优,熟悉Spark-Streaming, Kafka, Zookeeper的使用。
\item 熟练使用Drill, Pandas进行数据分析与报表制作,熟悉Drill SQL性能优化。
\item 熟悉使用MySql, Redis数据库系统,以及OLAP数据库Druid的使用。
\item 熟悉Android开发,熟悉Tornado,Netty网络编程框架,以及SqlAlchemy,MyBatis数据库框架的使用。
\item 熟悉使用Git, Docker, Gradle, Vim等效率工具。
\end{itemize}

% - Education
\section*{教育与培训}

\noindent\begin{tabular}{rl}
2005年9月 -- 2009年6月: & 南京工业大学,建筑环境与设备工程学士学位。
\end{tabular}

% - work experience
\section*{工作经验}

\noindent\begin{tabular}{L|l}
2010.7. -- 2014.9. & 
\parbox
    {\parlength}
    {
        {\bf 资深软件测试工程师} \\
        在NCS, IBM从事软件测试工作,拥有丰富的软件测试经验。在NCS期间曾两次获得客户颁发优质贡献证书。
        \begin{itemize}
        \item 参与或主导测试项目,涉及测试规划,测试设计,测试执行,以高责任心确保测试项目的质量周期完善可控。
        \item 拥有丰富的自动化与性能测试经验,多次参与自动化测试框架的设计与编写,拥有良好的编码能力和技术调研能力,能为客户提供高度针对性的方案与演示,多次为公司争取到回头客项目。
        \end{itemize}
    }
\end{tabular}\\\\

\noindent\begin{tabular}{L|l}
2014.10. -- 今 & 
\parbox
    {\parlength}
    {
        {\bf 上海触宝科技,大数据,全栈} \\
        以高级开发测试入职。先后主导并参与客户端部分重构与优化,参与后台服务开发,主导产品大数据分析,参与产品数据仓库建设及可视化工作。
        \begin{itemize}
            \item 重构Android测试框架,使用树来构建页面关系,实现基于名字的多页面自动跳转,提高了自动化脚本执行的稳定性;实现了基于YAML的对象统一管理框架,并编写了很多对象动态查找帮助方法,提高了脚本编写效率。
            \item 主导Android客户端的后台耗电优化工作。在此过程中基于AlarmManager实现了后台任务调度框架建设,形成一个统一管理的,支持灵活策略(如充电检测和WiFi检测运行)的任务调度框架。
            \item 独立重构了Android客户端的双卡识别引擎,期间调研了超过20个不同品牌的机型,建立了新的可自动识别的模型,并独立开发了双卡云数据同步服务,该模型能够自动识别市面上超过10000种不同厂商与型号的机型。
            \item 主导产品数据分析与报表,使用MapReduce, pig, spark, pandas等工具进行数据处理与分析,以Python进行任务监控与报表生成发送。除了常规报表之外,还先后帮助建立用户主线行为模型,用户注册漏斗模型,流失主因分析模型,用户通话分布模型,帮助提升了产品的注册率与留存率。
            \item 主导触宝电话的大数据A/B实验项目,基于Tornado+MySql开发了A/B实验的后台模块,提供模块间服务调用与客户端RESTful API支持,使用Sqoop任务将数据导入数据仓库,满足产品团队对于A/B实验的数据分析需求。
            \item 推动并参与产品的数据仓库建设,对产品数据进行归类梳理,使用Sqoop, MapReduce, Pig, Spark等大数据工具编写ETL任务将每日数据清洗进入数据仓库,同时组织编写基于Apache Drill的使用文档及数据文档,提供数据使用培训,大大提升了组内成员的数据使用意愿和使用能力。
            \item 使用Superset和Druid进行数据的可视化工作。使用Drill和Spark对数据进行计算与聚合,使用Python生成Druid导入文件并完成导入操作。对Superset进行二次开发(如支持模糊匹配筛选等)以提高用户体验。
        \end{itemize}

    }
\end{tabular}\\\\

\end{document}