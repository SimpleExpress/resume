% XeLaTex
% document type
\documentclass{ctexart}

% preamble
% - packages
\usepackage{array, titlesec, calc, fancyhdr}
\usepackage[margin=3cm]{geometry}
\usepackage[colorlinks=true, urlcolor=blue]{hyperref}

% - commands

\newcolumntype{L}{>{\raggedleft\arraybackslash}p{2cm}}
\newcolumntype{R}{>{\raggedleft\arraybackslash}p{3.5cm}}
\newcommand{\parlength}{\textwidth-2.5cm}

\fancypagestyle{onestyle}
{
    \fancyhead{}
    \fancyfoot[CO]{\thepage}
    \renewcommand{\headrulewidth}{0pt}
    \renewcommand{\footrulewidth}{0pt}
}

\titleformat
    {\section}
    {\normalfont\LARGE\bfseries}
    {\thesection}
    {1em}
    {}
    [{\titlerule[0.7pt]}]

\pagestyle{onestyle}

% - title
\author{}
\title{\vspace{-1cm}曹建明}
\date{}

% document
\begin{document}
% \maketitle


% - personal info
\section*{个人信息}

\noindent\begin{tabular}{rl}
姓名: & 曹建明\\
手机: & +86 155 0513 9159\\
邮箱: & \href{mailto:glide.ming@gmail.com}{glide.ming@gmail.com}\\
{\sc GitHub:} & \href{https://github.com/SimpleExpress}{https://github.com/SimpleExpress}\\
{\sc Readings:} & \href{http://book.douban.com/people/37889167/}{http://book.douban.com/people/37889167/}

\end{tabular}

% - Skill
\section*{职业技能}

\begin{itemize}
\item 有较强的沟通协调能力、学习调研能力及推动能力,能够独立推动项目顺利进行。
\item 熟练使用(5年以上)Python和Java, 熟悉Scala编程语言。
\item 熟练MapReduce, Pig, Spark任务代码编写,了解常用优化技巧。熟悉Spark-Streaming, Kafka, Zookeeper的使用。
\item 熟练使用Drill, Pandas进行数据分析与报表制作,了解Drill性能优化。
\item 熟悉MySql, Redis常用数据库系统,以及OLAP数据库Druid的使用。
\item 有Android开发经验,以及基于Tornado, Flask的服务器开发经验。
\item 熟悉使用Git, Docker, Gradle, Vim等效率工具。
\end{itemize}

% - Education
\section*{教育与培训}

\noindent\begin{tabular}{rl}
2005年9月 -- 2009年6月: & 南京工业大学,建筑环境与设备工程学士学位。
\end{tabular}

% - work experience
\section*{工作经验}

\noindent\begin{tabular}{L|l}
2010.7. -- 2014.9. & 
\parbox
    {\parlength}
    {
        {\bf 资深软件测试工程师} \\
        在NCS, IBM从事软件测试工作,拥有丰富的软件测试经验。在NCS期间曾两次获得客户颁发优质贡献证书。
        \begin{itemize}
        \item 参与或主导测试项目,涉及测试规划,测试设计,测试执行,确保测试项目的质量周期完善可控。
        \item 拥有丰富的自动化与性能测试经验,多次参与自动化测试框架的编写,拥有良好的编码能力和技术调研能力。
        \end{itemize}
    }
\end{tabular}\\\\

\noindent\begin{tabular}{L|l}
2014.10. -- 今 & 
\parbox
    {\parlength}
    {
        {\bf 上海触宝科技,大数据,客户端,服务器,开发测试} \\
        以高级开发测试入职。先后主导并参与客户端部分重构与优化,参与后台服务开发,主导产品大数据分析,参与产品数据仓库建设及可视化工作。
        \begin{itemize}
        \item 测试与开发工作(不到1年)。
        		\begin{itemize}
	        \item 部分重构已有的Android自动化测试框架,调研新的自动化/白盒测试工具与技术,面向开发人员及初级测试人员提供测试相关的分享培训。
	        \item 主导Android客户端的后台耗电优化工作。在此过程中完成了后台任务调度框架建设,形成一个统一管理的,支持灵活策略(如充电和WiFi环境优化检测)的任务调度框架。
        		\item 独立重构了Android客户端的双卡识别引擎,建立新的识别模型,极大提升了自动识别能力,并负责搭建了双卡云数据同步服务,进一步提高了识别的稳定性与可控性,使触宝电话的双卡识别能力得到大幅提升,并远领先于同类竞品。
		\end{itemize}
        \item 触宝电话产品数据工作(近1年半)。
        		\begin{itemize}
		\item 数据分析与报表。使用pig, spark, pandas等大数据工具进行数据处理分析与报表生成。先后与产品经理合作建立用户主线行为模型,用户注册漏斗模型,流失主因分析模型,完成数据埋点及后续处理自动化流程。帮助产品定位问题与优化,提升产品的注册率与留存率。
		\item 主导触宝电话的大数据A/B实验项目,基于Tornado+MySql开发了A/B实验的后台模块,提供模块间服务调用与客户端RESTful API支持,使用Sqoop任务将数据导入数据仓库,满足产品团队对于A/B实验的数据分析需求。
		\item 参与产品的数据仓库建设,对产品数据进行归类梳理,使用Sqoop, MapReduce, Pig, Spark等大数据工具编写ETL任务将每日数据清洗进入数据仓库,同时组织编写基于Apache Drill的使用文档,提供数据使用培训,大大提升了组内成员的数据使用意愿和使用能力。
		\item 调研开源可视化项目Superset,并结合Druid进行数据的可视化工作,对Superset进行优化与二次开发以满足产品团队的使用需求。期间编写了很多框架性质的代码,提高了其他成员从数据源选定,数据清洗,Druid导入文件生成,到最后数据导入Druid这一流程的效率。
		\item 分析总结了一套满足大部分常规需求埋点方式,并独立编写了双平台native和WebView的实现及后端数据处理作业,自动化这部分数据的清洗入库和可视化生成。极大地方便了开发人员的数据分析需求。
		\end{itemize}
        \end{itemize}
    }
\end{tabular}\\\\

\end{document}